\documentclass{article}
\usepackage{amsmath}
\usepackage{amsfonts}
\usepackage{amssymb}
\usepackage{graphicx}
\usepackage{hyperref}
\usepackage{xcolor}
\usepackage{listings}



\lstdefinestyle{CStyle}{
	language=C,
	backgroundcolor=\color{lightgray},   % 背景颜色
	basicstyle=\ttfamily\footnotesize,    % 基本字体
	keywordstyle=\color{blue},            % 关键字颜色
	commentstyle=\color{green!70!black},  % 注释颜色
	stringstyle=\color{red},               % 字符串颜色
	numbers=left,                         % 行号位置
	numberstyle=\tiny\color{gray},        % 行号样式
	stepnumber=1,                         % 行号步长
	numbersep=5pt,                        % 行号与代码间距
	tabsize=4,                            % 制表符宽度
	breaklines=true,                      % 自动换行
	showspaces=false,                     % 不显示空格
	showstringspaces=false,               % 不显示字符串中的空格
	escapeinside={(*@}{@*)}               % 自定义转义
}
\title{1.6.8 Exercises for Section 1.6}
\begin{document}
\maketitle
\section{Exercise 1.6.1}
For the block-structured C code of Fig. 1.13(a), indicate the values assigned to w, x, y and z.
\begin{lstlisting}[style=CStyle]
	int w, x, y, z;
	int i = 4; int j = 5;
	{ int j = 7;
	  i = 6;
	  w = i + j;
    }
    x = i + j;
    {  int i = 8;
       y = i + j;
    }
    z = i + j;
\end{lstlisting}
\subsection{answer:}
\begin{enumerate}
	\item w is assigned to 13
	\item x is assigned to 9
	\item y is assigned to 13
	\item z is assigned to 9
\end{enumerate}

\newpage
\section{Exercise 1.6.2}
Repeat Exercise 1.6.1 for the code of Fig. 1.13(b)
\begin{lstlisting}[style=CStyle]
	int w, x, y, z;
	int i = 3; int j = 4;
	{	int i = 5;
		w = i + j;
	}
	x = i + j;
	{	int j = 6;
		i = 7;
		y = i + j;
	}
	z = i + j;
\end{lstlisting}

\subsection{Answer:}
\begin{enumerate}
	\item w is assigned 9
	\item x is assigned 7
	\item y is assigned 13
	\item z is assigned 11
\end{enumerate}

\newpage
\section{Exercise 1.6.3}
For the block-structured code of Fig.1.14, assuming the usual static scoping of declarations, give the scope for each twelve declarations.
\begin{lstlisting}[style=CStyle]
	{	int w, x, y, z;		/* Block B1 */
		{	int x, z;		/* Block B2 */
			{	int w, z;	/* Block B3 */}
		}
		{	int w, x;		/* Block B4 */
			{  int y, z		/* Block B5 */ }
		}
	}
\end{lstlisting}
\subsection{Answer:}
\begin{enumerate}
	\item w(B1):B1, B2
	\item x(B1):B1
	\item y(B1):B1, B2, B3, B4
	\item z(B1):B1, B4
	\item x(B2):B2, B3
	\item z(B2):B2
	\item w(B3):B3
	\item z(B3):B3
	\item w(B4):B4, B5
	\item x(B4):B4, B5
	\item y(B5):B5
	\item z(B5):B5
	
\end{enumerate}

\section{Exercise 1.6.4}
What is printed by the following C code?
\begin{lstlisting}[style=CStyle]
#define	a	(x+1)
int x = 2;
void b() { x = a; printf("%d\n", x); }
void c() { int x = 1; printf("%d\n", a)}
void main() { b(); c();}
\end{lstlisting}
\subsection{Answer:}
3\newline
1
\end{document}